\documentclass[10.5pt]{article}

% Spanish characters
\usepackage[utf8]{inputenc}
\usepackage[T1]{fontenc}
% French display
\usepackage[english,spanish]{babel}

\usepackage{lastpage}
%Esto me permite usar el comando "\pageref{LastPage}" en el footer.
\renewcommand{\baselinestretch}{1.6}
% Esto controla el interlineado o espaciado!!!
\usepackage{color}
%\newcommand{\red}[1]{{\color{red} #1}}
\newcommand{\red}[1]{{\color{black} #1}}

%The following packages are relics, but I don't want to remove them just in case:
\usepackage{amsmath}
\usepackage{array}
\usepackage{latexsym}
\usepackage{amsfonts}
\usepackage{amsthm}
\usepackage{cite}
\usepackage{setspace}
\usepackage{amssymb}
\usepackage{hyperref}

\usepackage{multicol}
\usepackage{color}
%\usepackage{minipage}

\usepackage{graphicx} % Required for including images
\graphicspath{{figures/}} % Location of the graphics files
\usepackage[font=small,labelfont=bf]{caption} % Required for specifying captions to tables and figures

%The defaults margins are huge, so I'll customize it:
\oddsidemargin  -0.0 in
\textwidth 6.5 in
\textheight 8.7 in
\addtolength{\voffset}{-1cm}

%%%%%%%%%%%%%%%%%%%%%%%% HEADER AND FOOTER %%%%%%%%%%%%%%%%%%%%
\usepackage{fancyhdr}
\pagestyle{fancy}

\fancyhead[L]{Introducción}
\fancyhead[R]{Jos\'{e} David Ruiz \'{A}lvarez}
\fancyhead[C]{}
\fancyfoot[C]{\thepage /\pageref{LastPage}}

\newlength\FHoffset
\setlength\FHoffset{0cm}

\addtolength\headwidth{2\FHoffset}
\fancyheadoffset{\FHoffset}

\newlength\FHleft
\newlength\FHright

\setlength\FHleft{1cm}
\setlength\FHright{1cm}

\thispagestyle{empty}
%%%%%%%%%%%%%%%%%%%%%%%% HEADER AND FOOTER %%%%%%%%%%%%%%%%%%%%



\begin{document}

\noindent
\begin{minipage}[b]{0.75\linewidth}
{\LARGE\bf Introducción}\\ %[1mm]
\large{Jos\'{e} David Ruiz \'{A}lvarez} \\
\small{\href{mailto:josed.ruiz@udea.edu.co}{josed.ruiz@udea.edu.co}} \\ %[3mm]
\normalsize{Instituto de Física, Facultad de Ciencias Exactas y Naturales} \\%[3mm]
\normalsize{\bf Universidad de Antioquia} \\[8mm]
\today %\\[4mm]
\end{minipage}%

\section{Contenido}

Dos grandes vertientes en t\'{e}rminos de problemas f\'{i}sicos: 
\begin{itemize}
\item Lenguajes de programación, paquetes, entre otros.
\item Solución numérica de ecuaciones diferenciales
\item Técnicas de Monte Carlo
\item Análisis estadístico de datos
\end{itemize}

\section{Evaluación}

\begin{itemize}
\item 30\% seguimiento y tareas: Problemas cortos y ejercicios de programación. (Dividido en dos seguimientos del 15\%)
\item 20\% proyecto: Problema a resolver en grupos.
\item 50\%, dos parciales del 25\%. 
\end{itemize}

\begin{itemize}
\item Seguimiento 1: Agosto 6 a Septimembre 19.
\item Taller preparatorio Parcial 1: Septiembre 24.
\item Parcial 1: Septiembre 26.
\item Seguimiento 2: Octubre 1 a Noviembre 5.
\item Taller preparatorio Parcial 2: Noviembre 7.
\item Parcial 2: Noviembre 12.
\item Proyecto: Noviembre 14 al 26.
\item Presentación proyectos: Noviembre 28.
\end{itemize}

\section{Evaluación diagnóstica}

\url{https://docs.google.com/forms/d/e/1FAIpQLSf7Bq8m7FjbkGYRuGSj__oLrZ8t9UurSi0MS6OqpUYhNz2zwA/viewform?usp=sf_link}

\section{Repositorio del curso}

\url{https://github.com/jotadram6/CUFICO_2019_2}

\end{document}

%%% Local Variables:
%%%   mode: latex
%%%   mode: flyspell
%%%   ispell-local-dictionary: "spanish"
%%% End:
